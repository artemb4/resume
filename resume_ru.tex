%-------------------------
% Resume in Latex
% Author : Artem Bocharov
% License : MIT
%------------------------

\documentclass[letter-paper,10pt]{article}

\usepackage{makecell}
\usepackage[link=off]{phonenumbers}
\usepackage{ragged2e}
\usepackage[T2A]{fontenc}
\usepackage[utf8]{inputenc}  % зависит от кодировки Вашего документа
\usepackage[english,russian]{babel}
\usepackage{latexsym}
\usepackage[empty]{fullpage}
\usepackage{titlesec}
\usepackage{marvosym}
\usepackage[usenames,dvipsnames]{color}
\usepackage{verbatim}
\usepackage{enumitem}
\usepackage[pdftex]{hyperref}
\usepackage{fancyhdr}


\pagestyle{fancy}
\fancyhf{} % clear all header fields
\fancyfoot{} % clear all footer fields
\renewcommand{\headrulewidth}{0pt}
\renewcommand{\footrulewidth}{0pt}
\usepackage[margin=0.3in]{geometry}

% Adjust margins
\addtolength{\oddsidemargin}{-0.0in}
\addtolength{\evensidemargin}{-0.0in}
\addtolength{\textwidth}{0in}
\addtolength{\topmargin}{10pt}
\addtolength{\textheight}{0.0in}

\urlstyle{same}

\usepackage{xcolor} % http://ctan.org/pkg/xcolor
\usepackage{hyperref} % http://ctan.org/pkg/hyperref
\hypersetup{
  colorlinks=true,
  linkcolor=blue!50!red,
  linkbordercolor=red,
  urlcolor=black!70!black,
  pdfnewwindow=true
}

\raggedbottom
\raggedright
\setlength{\tabcolsep}{0in}

% Sections formatting
\titleformat{\section}{
  \vspace{-10pt}\scshape\raggedright\large
}{}{0em}{}[\color{black}\titlerule \vspace{-7pt}]

%-------------------------
% Custom commands
\def \ifempty#1{\def\temp{#1} \ifx\temp\empty }

\newcommand{\resumeItem}[2]{
  \item\small{
  	\ifempty{#1}#2\else\textbf{#1}{: #2 \vspace{-2pt}}\fi
  }
}

\usepackage[dvipsnames]{xcolor}
\definecolor{mygray}{gray}{0}
\usepackage{fancybox}

\usepackage{lmodern}
\usepackage{tikz}

% Style definition
\tikzset{rndblock/.style={rounded corners,rectangle,draw,outer sep=0pt}}

% Command Definition
% 1 optional to customize the aspect, 2 mandatory: text to be framed
\newcommand{\tframed}[2][]{\tikz[baseline=(h.base)]\node[rndblock,#1] (h) {\color{black}{#2}};}

\newcommand*{\mystrut}{\rule[-0.2\baselineskip]{0pt}{0.8\baselineskip}}
\newcommand{\skill}[1]{\tframed[lightgray]{\mystrut#1}}


\newcommand{\resumeSubheading}[4]{
  \item
    \begin{tabular*}{0.97\textwidth}{l@{\extracolsep{\fill}}r}
      \vspace{-12pt}\textbf{#3} & \textcolor{mygray}{\textit{\small #2}} \\
      \textit{\small#1} & \textcolor{mygray}{\textit{\small #4}} \\
    \end{tabular*}\vspace{-5pt}
}

\newcommand{\resumeExpSubheading}[5]{
  \vspace{3pt}
  \item
    \begin{tabular*}{0.97\textwidth}{l@{\extracolsep{\fill}}r}
      \vspace{2pt} \textbf{#1}  & \textcolor{mygray}{\small #2} \\
      \textit{#3} & \textcolor{mygray}{\textit{\small #4}} \\
      {\scriptsize#5}
    \end{tabular*}\vspace{3pt}
}

\newcommand{\resumeExpJointSubheading}[5]{
  \vspace{-14pt}
  \item
    \begin{tabular*}{0.97\textwidth}{l@{\extracolsep{\fill}}r}
      \vspace{2pt} \textbf{#1}  & \textcolor{mygray}{\small #2} \\
      \textit{#3} & \textcolor{mygray}{\textit{\small #4}} \\
      {\scriptsize#5}
    \end{tabular*}\vspace{3pt}
}

\newcommand{\resumeProjSubheading}[5]{
  \vspace{-10pt}\item
    \begin{tabular*}{0.97\textwidth}{l@{\extracolsep{\fill}}r}
      \vspace{2pt} \textbf{#1}  & \textcolor{mygray}{\small #2} \\
      \textbf{#3} & \textcolor{mygray}{\textit{\small #4}} \\
      {\scriptsize#5}
    \end{tabular*}\vspace{3pt}
}

\newcommand{\resumeSubItem}[2]{\resumeItem{#1}{#2}\vspace{-4pt}}

\renewcommand{\labelitemii}{$\circ$}

\newcommand{\resumeSubHeadingListStart}{\begin{itemize}[leftmargin=*]}
\newcommand{\resumeSubHeadingListEnd}{\end{itemize}}
\newcommand{\resumeItemListStart}{\begin{itemize}[leftmargin=0.2in]}
\newcommand{\resumeItemListEnd}{\end{itemize}\vspace{-5pt}}

\usepackage{changepage}
\newcommand{\resumeDesc}[1]{\begin{adjustwidth}{5pt}{0pt}\vspace{-2pt}{#1}\end{adjustwidth}}

\newcommand{\ExternalLink}{
    \tikz[x=1.2ex, y=1.2ex, baseline=-0.05ex]{
        \begin{scope}[x=1ex, y=1ex]
            \clip (-0.1,-0.1)
                --++ (-0, 1.2)
                --++ (0.6, 0)
                --++ (0, -0.6)
                --++ (0.6, 0)
                --++ (0, -1);
            \path[draw,
                line width = 0.5,
                rounded corners=0.5]
                (0,0) rectangle (1,1);
        \end{scope}
        \path[draw, line width = 0.5] (0.5, 0.5)
            -- (1, 1);
        \path[draw, line width = 0.5] (0.6, 1)
            -- (1, 1) -- (1, 0.6);
        }
    }

\definecolor{Blue1}{HTML}{4D4EDC}
\newcommand{\MYhref}[3][Blue1]{\href{#2}{\color{#1}{#3}}}

%-------------------------------------------
%%%%%%  CV STARTS HERE  %%%%%%%%%%%%%%%%%%%%%%%%%%%%


\begin{document}
%----------HEADING-----------------

\begin{center}\textbf{\Large Артём Бочаров}\end{center}
\vspace{-12pt}
\begin{center}
Email: \MYhref{mailto:bo4arvoartyom@yandex.ru}{bo4arvoartyom@yandex.ru} \quad
GitHub: \MYhref{https://github.com/artemb4}{artemb4}
Telegram: @Chapaev\_task\_is\_unsolved
\end{center}

%-----------RESUME-----------------
\vspace{-10pt}
\section{Резюме}
\resumeSubHeadingListStart
\justifying
Backend разработчик со стажем более 3 лет. \\
Специализируюсь на продуктовых задачах. \\
Помогаю вести проекты.
\resumeSubHeadingListEnd

%-----------EXPERIENCE-----------------

\vspace{-5pt}
\section{Опыт}
\justifying
  \resumeSubHeadingListStart

% Yandex
    \resumeExpSubheading
      {\href{https://yandex.ru/maps}{Яндекс Лавка\ExternalLink}}{Москва, Россия}
      {Backend-разработчик (С++) в Яндексе}{Октябрь 2022 --- Настоящее время}
      {\skill{C++} \skill{Python} \skill{PostgreSQL} \skill{boost} \skill{GTest}}
      \resumeDesc{
      \begin{itemize}
          \item Добавил долги в критических ситуациях (например, если проблемы с оплатами на стороне банка)
          \item Интегрировался с антифродом
          \item Автоудаление чеков самозанятых
          \item Упростил добавление скидок
          \item Подробный чек на экране чекаута/трекинга
      \end{itemize}}

% IP
    \resumeExpSubheading
      {ИП Иванов}{Москва, Россия}
      {C\#-разработчик по разработке средства создания серверов для игры, схожей с Roblox}{Август 2021 --- Октябрь 2022}
      {\skill{C\#} \skill{sandbox} \skill{steam}}
      \resumeDesc{
      \begin{itemize}
        \item Добавил несколько игровых механик, включая использование оружия и возможность передвигаться на машинах
        \item Создал базовый сервер для отладки функционала
      \end{itemize}}


%--------SKILLS------------
\section{Навыки}
    \resumeSubHeadingListStart
    \begin{tabular}{ll}
\textbf{Languages:} & \quad C++, C, Python, Java, C\#, SQL, PostgreSQL \\
\textbf{Tools:} & \quad Git, Docker, Grafana, Elasticsearch, FastAPI, Charles
\end{tabular}
\resumeSubHeadingListEnd

%-----------EDUCATION-----------------
\section{Образование}
    \resumeSubHeadingListStart
    \resumeSubheading
    {}{}
    {\href{https://spb.hse.ru/ba/appmath/}{Бакалавр НИУ ВШЭ СПБ 2021-2022 \ExternalLink}}{}{
    \vspace{5pt}
    \newline{\textbf{Направление:} Прикладная математика и информатика/}
    \newline{\textbf{GPA:} 6.86}
    }
    \\
    {\href{https://abit.itmo.ru/program/bachelor/software_engineering}{Бакалавр ИТМО 2022-2026 \ExternalLink}}{}{
    \vspace{5pt}
    \newline{\textbf{Направление:} Разработка программного обеспечения/}
    \newline{\textbf{GPA:} 4.1}
    }
\resumeSubHeadingListEnd


\end{document}

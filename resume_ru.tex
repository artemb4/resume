%-------------------------
% Resume in Latex
% Author : Daniel Pustotin
% License : MIT
%------------------------

\documentclass[letter-paper,10pt]{article}

\usepackage{makecell}
\usepackage[link=off]{phonenumbers}
\usepackage{ragged2e}
\usepackage[T2A]{fontenc}
\usepackage[utf8]{inputenc}  % зависит от кодировки Вашего документа
\usepackage[english,russian]{babel}
\usepackage{latexsym}
\usepackage[empty]{fullpage}
\usepackage{titlesec}
\usepackage{marvosym}
\usepackage[usenames,dvipsnames]{color}
\usepackage{verbatim}
\usepackage{enumitem}
\usepackage[pdftex]{hyperref}
\usepackage{fancyhdr}


\pagestyle{fancy}
\fancyhf{} % clear all header fields
\fancyfoot{} % clear all footer fields
\renewcommand{\headrulewidth}{0pt}
\renewcommand{\footrulewidth}{0pt}
\usepackage[margin=0.3in]{geometry}

% Adjust margins
\addtolength{\oddsidemargin}{-0.0in}
\addtolength{\evensidemargin}{-0.0in}
\addtolength{\textwidth}{0in}
\addtolength{\topmargin}{10pt}
\addtolength{\textheight}{0.0in}

\urlstyle{same}

\usepackage{xcolor} % http://ctan.org/pkg/xcolor
\usepackage{hyperref} % http://ctan.org/pkg/hyperref
\hypersetup{
  colorlinks=true,
  linkcolor=blue!50!red,
  linkbordercolor=red,
  urlcolor=black!70!black,
  pdfnewwindow=true
}

\raggedbottom
\raggedright
\setlength{\tabcolsep}{0in}

% Sections formatting
\titleformat{\section}{
  \vspace{-10pt}\scshape\raggedright\large
}{}{0em}{}[\color{black}\titlerule \vspace{-7pt}]

%-------------------------
% Custom commands
\def \ifempty#1{\def\temp{#1} \ifx\temp\empty }

\newcommand{\resumeItem}[2]{
  \item\small{
  	\ifempty{#1}#2\else\textbf{#1}{: #2 \vspace{-2pt}}\fi
  }
}

\usepackage[dvipsnames]{xcolor}
\definecolor{mygray}{gray}{0}
\usepackage{fancybox}

\usepackage{lmodern}
\usepackage{tikz}

% Style definition
\tikzset{rndblock/.style={rounded corners,rectangle,draw,outer sep=0pt}}

% Command Definition
% 1 optional to customize the aspect, 2 mandatory: text to be framed
\newcommand{\tframed}[2][]{\tikz[baseline=(h.base)]\node[rndblock,#1] (h) {\color{black}{#2}};}

\newcommand*{\mystrut}{\rule[-0.2\baselineskip]{0pt}{0.8\baselineskip}}
\newcommand{\skill}[1]{\tframed[lightgray]{\mystrut#1}}


\newcommand{\resumeSubheading}[4]{
  \item
    \begin{tabular*}{0.97\textwidth}{l@{\extracolsep{\fill}}r}
      \vspace{-12pt}\textbf{#3} & \textcolor{mygray}{\textit{\small #2}} \\
      \textit{\small#1} & \textcolor{mygray}{\textit{\small #4}} \\
    \end{tabular*}\vspace{-5pt}
}

\newcommand{\resumeExpSubheading}[5]{
  \vspace{3pt}
  \item
    \begin{tabular*}{0.97\textwidth}{l@{\extracolsep{\fill}}r}
      \vspace{2pt} \textbf{#1}  & \textcolor{mygray}{\small #2} \\
      \textit{#3} & \textcolor{mygray}{\textit{\small #4}} \\
      {\scriptsize#5}
    \end{tabular*}\vspace{3pt}
}

\newcommand{\resumeExpJointSubheading}[5]{
  \vspace{-14pt}
  \item
    \begin{tabular*}{0.97\textwidth}{l@{\extracolsep{\fill}}r}
      \vspace{2pt} \textbf{#1}  & \textcolor{mygray}{\small #2} \\
      \textit{#3} & \textcolor{mygray}{\textit{\small #4}} \\
      {\scriptsize#5}
    \end{tabular*}\vspace{3pt}
}

\newcommand{\resumeProjSubheading}[5]{
  \vspace{-10pt}\item
    \begin{tabular*}{0.97\textwidth}{l@{\extracolsep{\fill}}r}
      \vspace{2pt} \textbf{#1}  & \textcolor{mygray}{\small #2} \\
      \textbf{#3} & \textcolor{mygray}{\textit{\small #4}} \\
      {\scriptsize#5}
    \end{tabular*}\vspace{3pt}
}

\newcommand{\resumeSubItem}[2]{\resumeItem{#1}{#2}\vspace{-4pt}}

\renewcommand{\labelitemii}{$\circ$}

\newcommand{\resumeSubHeadingListStart}{\begin{itemize}[leftmargin=*]}
\newcommand{\resumeSubHeadingListEnd}{\end{itemize}}
\newcommand{\resumeItemListStart}{\begin{itemize}[leftmargin=0.2in]}
\newcommand{\resumeItemListEnd}{\end{itemize}\vspace{-5pt}}

\usepackage{changepage}
\newcommand{\resumeDesc}[1]{\begin{adjustwidth}{5pt}{0pt}\vspace{-2pt}{#1}\end{adjustwidth}}

\newcommand{\ExternalLink}{
    \tikz[x=1.2ex, y=1.2ex, baseline=-0.05ex]{
        \begin{scope}[x=1ex, y=1ex]
            \clip (-0.1,-0.1)
                --++ (-0, 1.2)
                --++ (0.6, 0)
                --++ (0, -0.6)
                --++ (0.6, 0)
                --++ (0, -1);
            \path[draw,
                line width = 0.5,
                rounded corners=0.5]
                (0,0) rectangle (1,1);
        \end{scope}
        \path[draw, line width = 0.5] (0.5, 0.5)
            -- (1, 1);
        \path[draw, line width = 0.5] (0.6, 1)
            -- (1, 1) -- (1, 0.6);
        }
    }

\definecolor{Blue1}{HTML}{4D4EDC}
\newcommand{\MYhref}[3][Blue1]{\href{#2}{\color{#1}{#3}}}

%-------------------------------------------
%%%%%%  CV STARTS HERE  %%%%%%%%%%%%%%%%%%%%%%%%%%%%


\begin{document}
%----------HEADING-----------------

\begin{center}\textbf{\Large Даниил Пустотин}\end{center}
\vspace{-12pt}
\begin{center}
Email: \MYhref{mailto:daniel.pustotin@gmail.com}{daniel.pustotin@gmail.com} \quad
LinkedIn: \MYhref{http://linkedin.com/in/daniel-pustotin}{daniel-pustotin} \quad
GitHub: \MYhref{https://github.com/heartsker}{heartsker}
\end{center}

%-----------RESUME-----------------
\vspace{-10pt}
\section{Резюме}
\resumeSubHeadingListStart
\justifying
iOS разработчик со стажем 5 лет\\
Специализируюсь на продуктовых задачах\\
Веду проекты и занимаюсь техническим менеджментом разработки
\resumeSubHeadingListEnd

%-----------EXPERIENCE-----------------

\vspace{-5pt}
\section{Опыт}
\justifying
  \resumeSubHeadingListStart

% Alfabank
    \resumeExpSubheading
      {\href{https://alfabank.ru/sme/rko/abm}{Альфа-Банк\ExternalLink}}{Москва, Россия}
      {Senior iOS Developer в команде разработки приложения Альфа-Бизнес}{Февраль 2024 --- Настоящее время}
      {\skill{UIKit} \skill{RxSwift} \skill{Kotlin Multiplatform} \skill{MVVM} \skill{Dependency Injection} \skill{Project Management}}
      \resumeDesc{
      \begin{itemize}
          \item Добавил чаты с клиентскими менеджерами и поддержкой
          \item Добавил групповые чаты для владельцев бизнесов
      \end{itemize}}

% Yandex
    \resumeExpSubheading
      {\href{https://yandex.ru/maps}{Яндекс Карты\ExternalLink}}{Москва, Россия}
      {Middle iOS Developer в продуктовой команде разработки Яндекс Карт}{Август 2022 --- Февраль 2024}
      {\skill{UIKit} \skill{RxSwift} \skill{Kotlin Multiplatform} \skill{MVVM} \skill{Dependency Injection} \skill{Project Management}}
      \resumeDesc{
      \begin{itemize}
          \item Добавил сервис Музыки в Карты
          \item Добавил маркировку всех форматов рекламы
          \item Развивал сервисы Заправок и Парковок
          \item Разрабатывал документацию и примеры для публичной библиотеки \MYhref{https://yandex.ru/dev/mapkit/doc/en}{Yandex MapKit}
          \item Учавствовал в адаптировании интерфейса для RTL-языков
      \end{itemize}}

% Tinkoff
    \resumeExpSubheading
    {\href{https://www.tinkoff.ru}{Тинькофф\ExternalLink}}{Москва, Россия}
    {iOS Developer Intern в команде Мобильного Банка в Тинькофф}{Февраль 2022 --- Май 2022}
    {\skill{UIKit} \skill{Swinject} \skill{Unit/UI testing}}
    \resumeDesc{
    \begin{itemize}
        \item Добавил модуль определителя звонков
        \item Занимался продуктовыми задачами в приложении Мобильного Банка
        \item Исправлял критические проблемы и уязвимости
    \end{itemize}}

% Skillbox
    \resumeExpSubheading
    {\href{https://skillbox.ru}{Skillbox\ExternalLink}}{Удаленно}
    {Создатель и ментор курсов по iOS-разработке}{Июнь 2021 --- Настоящее время}
    {\skill{LaTeX} \skill{GitHub} \skill{Project management}}
    \resumeDesc{
    \begin{itemize}
        \item Создавал и ревьюил учебные задачи и методические материалы
        \item Выпустил более 5 потоков студентов по направлениям Swift разработка, SwiftUI, Разработка мобильных приложений
    \end{itemize}}

% GSSoft
    \resumeExpSubheading
    {\href{https://www.gs-soft.com/CMS/en}{GS-Soft\ExternalLink}}{Удаленно}
    {QA-инженер}{Май 2018 --- Июль 2021}
    {\skill{TypeScript} \skill{JavaScript} \skill{Angular} \skill{React JS} \skill{SVN}}
    \resumeDesc{
    \begin{itemize}
        \item Автоматизированное тестирование и QA-инжиниринг
        \item Написание различных видов тестов на TypeScript
    \end{itemize}}
    \resumeSubHeadingListEnd

%--------SKILLS------------
\section{Навыки}
    \resumeSubHeadingListStart
    \begin{tabular}{ll}
\textbf{Languages:} & \quad Swift, Kotlin, C++, C, Python\\
\textbf{Frameworks:} & \quad SwiftUI, UIKit, RxSwift, Swinject, Needle\\
\textbf{Tools:} & \quad Git, Docker, Prometheus, Yandex Metrica, FastAPI, Charles
\end{tabular}
\resumeSubHeadingListEnd

%-----------EDUCATION-----------------
\section{Образование}
    \resumeSubHeadingListStart
    \resumeSubheading
    {}{}
    {\href{https://hse.ru/ba/ami/}{Бакалавр НИУ ВШЭ 2020-2024 \ExternalLink}}{}{
    \vspace{5pt}
    \newline{\textbf{Направление:} Прикладная математика и информатика/}
    \newline{\textbf{Percentile:} 0.0}
    \newline{\textbf{GPA:} 8.24}
    \newline{\textbf{Мобильность:} Участник внутриуниверситетской мобильности с 2021 по 2024 год в НИУ ВШЭ, Факультет компьютерных наук, Москва}
    }
\resumeSubHeadingListEnd

%-----------AWARDS--------------
\section{Достижения}
\resumeSubHeadingListStart
\begin{tabular}{ll}
    \textbf{Google HashCode 2021} & \quad Top 30\% (worldwide)\\
    \textbf{Reply Challenge 2021} & \quad \#105 из 1923 (worldwide)\\
    \textbf{ICPC 2020} & \quad Полуфиналист \\
\end{tabular}
\resumeSubHeadingListEnd

\end{document}
